\section{Background of the Study}

Accidents which are external causes of death and injury is the 4th leading cause of death for all ages in the Philippines. Based on the statistics from 1975 to 2002, injury mortality rates have more than doubled from 19.1 in 1975 to 41.9 in 2003 per 100,000 population. One of the top five leading causes of death due to accidents in the Philippines for all ages is accidental falls. From the 4, 947 patients admitted to the Division of Trauma during a study period, 231 (4.7\%) deaths were recorded and 205 (88.7\%) of these were males. 135 (58.4\%) from death were victims of penetrating injuries, and 96 (41.6\%) had blunt trauma (vehicular injury in 66, falls in 15, mauling in 5, and other blunt injuries in 10). Intentional causes of injury (stab wound, gunshot wounds, and blunt assault) led to 151 (65\%) deaths, while unintentional causes (vehicular crashes and falls) caused 80 (35\%) deaths. \cite{Mar11} %(Marinas, Maddunba, Consunji, & Dela Paz Jr., 2011)

From worldwide statistics, 138 children, aged 0 to 18 years die daily due to falls. This translates to around 50,000 children dying each year due to accidental falls. It counts as the fifth leading cause of unintentional deaths of children below 14 years old in the Philippines. Falls were also found to be the leading cause of morbidity and lifelong disability among children. \cite{Cri14} %(Crisostomo, 2014)

From the Filipino elders in a nursing home facility and at the rehabilitation medicine out-patient department of the Philippine General Hospital (PGH), falls are considered one of the most serious health concerns encountered by the elderly. About 30 to 40\% of individuals living in the community aged 65 and above fall each year. These accidents are associated with increased morbidity and mortality, and as much as 20 to 30\% of those who fall suffer from serious hip fracture and head trauma. Current data show that falls comprise the single largest cause of death due to injury in the elderly. Recovery from falls is often poor because of restricted mobility and functional decline. Most falls have multiple causes, and are usually due to dynamic interplay of predisposing and precipitating factors. \cite{Gue} %(Guevarra & Evangelista, 2010)

\section{Prior Studies}

\begin{center}
{\scriptsize
\begin{tabularx}{\textwidth}{p{0.4\textwidth}|p{0.5\textwidth}}
\caption{Prior Studies} \label{Prior Studies} \\
\hline 
\hline 
\textbf{Existing Study} & 
\textbf{Description} \\ 
\hline 
\endfirsthead
\multicolumn{2}{c}%
{\textit{Continued from previous page}} \\
\hline
\hline 
\textbf{Existing Work} & 
\textbf{Lacking Approaches} \\ 
\hline 
\endhead
\hline 
\multicolumn{2}{r}{\textit{Continued on next page}} \\ 
\endfoot
\hline 
\endlastfoot
\hline

Design and Development of Fall Detector Using Fall Acceleration \cite{Sud} %(Sudarshan, Raveendra, Prasanna, \& Satyanarayana, 2013)
&
This study aims to design and develop a prototype of an electronic gadget which is used to detect fall among elderly and the patients who are prone to it. A triaxial accelerometer (adx1335) is used in this study to measure the change of acceleration in three axes which are integrated from the body posture. To study the tilt angle, sensors are placed on the lumbar region of the body. To reduce the false alarms, the acceleration values in each axis are compared twice with threshold and a delay of 20 secs between two comparisons. Values of the threshold voltage are selected by experimental methods. The algorithm is executed by microcontroller (PIC16F877A). A GPS receiver uses the location of fall which is programmed to track the subject continuously. 

On detection of fall, the device sends a text message through GSM modem, and communicates it to computer through ZigBee transceivers. The device can also be switched to alarm only if a text message is not required. The prototype developed is tested on many subjects and on volunteers who simulated fall. Out of 50 trials 96\% of accuracy is achieved with zero false alarms for daily activities like jogging, skipping, walking on stairs, and picking up objects. \\

\hline

Accurate, Fast Fall Detection Using Gyroscopes and Accelerometer-Derived Posture Information \cite{LiQ09} %(Li, Stankovic, Hanson, Barth, & Lach, 2009)
&
As prevalent methods only use accelerometers to isolate falls from activities of \acr{ADL}, certain fall-like activities such as sitting down quickly and jumping result in many false positives which make it difficult to distinguish real falls. In some other non-horizontal ending position like falls on the stairs, the use of body orientation is not very useful to detect falls.

This study has created a novel fall detection system using both accelerometers and gyroscopes. Human activities are divided into two categories: static postures and dynamic transitions. By using two tri-axial accelerometers at separate body locations, our system can recognize four kinds of static postures: standing, bending, sitting, and lying. Motions between these static postures are considered as dynamic transitions. Linear acceleration and angular velocity are measured to determine whether motion transitions are intentional. If the transition before a lying posture
is not intentional, a fall event is detected. The algorithm, coupled with accelerometers, reduces both false positives and false negatives, while improving fall detection accuracy. In addition, the solution features low computational cost and real-time response. 
  \\

\end{tabularx}
}
\end{center}

\section{Problem Statement}
\label{sec:statement}

Many simple fall related accidents that lead to death or irreversible damage due to slow dissemination or lack of knowledge of responders (e.g. family members, friends or rescuers), can be prevented if information regarding the incident can be sent immediately to a responder.

Fall incidents are enumerated into different circumstances that defines itself as accident. \gls{Fall} in this study is defined as an act of falling or collapsing or sudden uncontrollable descent or to suddenly go down onto the ground or towards the ground without intending to or by accident  \cite{Cam17}. %(Cambridge University Press, 2017).
It can also be defined as a sudden unintentional and uncontrollable descent which can also be classified as to stumble or to trip.  

This study would like to give a solution to accidents which can be prevented if knowledge or information regarding the fall incident can be identified and sent quickly. It also aims to give aid to the innovation of the past technologies that has been done in relation to these problems.

Many different studies and technologies are made from the past to detect fall, but these studies differ in accordance on how fall is being focused. The observation of Raul Igual et. al in their fall detection analysis trends states that, in the study of Li, the use of gyroscope and accelerometer in analyzation of fall types such as falling forward, backward and sideways is a success. Compared to the other technologies used to detect fall like image processing, the use of accelerometer is similar to this study. The advantage of this study over Li’s is that in her study, she located the sensors on the waist and chest, whereas this study will locate the sensors on the hip and on the thigh as based from the past studies these are better locations to achieve a higher success rate. \cite{Igu13} %(Igual, Medrano, & Plaza, 2013)

\section{Objectives}
\label{sec:objectives}

\subsection{General Objective(s)}
To design a device that can detect user’s motion and can send a notification to responder/s via \acr{GSM} when fall is detected.

\subsection{Specific Objectives}

\begin{enumerate}
	\item To develop a system that can analyze the duration and direction of the fall whether the user has been fallen backward, forward, or sideward with 80\% of accuracy upon detection.
	
	\item To create an algorithm that will store data collected from the user to a \acr{SD Card} that has user-friendly functions.
	
	\item To collect data that will be stored in the database from the user such as location and certain motions using \acr{GPS} and accelerometer respectively.
	
	\item To develop a system that will allow the device to send information to the responder/s.
	
	\item To develop an algorithm that can detect specific motions of the user such as standing, walking and lying.
\end{enumerate}

\section{Significance of the Study}

Imagine the advantages of having a device that can detect and acknowledge a person’s motion specifically when they unintentionally fell or tripped. As fall is one of the leading causes of death, it is an advantage to know how and when to help the victims after the incident.

This project aims to create a device that will innovate the detection of such incident. It also aims to notify certain responders when the said incident happened. By using this device, the goal to respond to fall related accidents will be achieved. It will be a solution to the fast-growing statistics of deaths, and other injuries that fall related accidents give.

Also, as the device is done, it can be used for further development of fall analyzer with the use of accelerometer. It can also be used in the medical field and through innovation, the detection and notification of a fallen patient can be recognized easily which may eventually lead to the prevention of fall related accidents.

\subsection{Scope}
\begin{enumerate}
	\item The device can only detect limited fall directions such as forward, backward and sideward.
	
	\item The device will gather the data for the motion and GPS coordinates in real time.
	
	\item This project will only require 30 people with an age range of 8 to 24 years old as test subjects. These subjects will perform activities for raw data gathering and would be translated into digital information eventually.
	
	\item A text-based database will be created to store the given data into the SD card. It could only store a limited amount of data so an “Automatic delete and save algorithm” will be implemented to save important data. The device will make a beeping sound to notify the user if it reached 90\% of its memory storage. It will automatically clear the data that has been saved and the unnecessary data will be deleted to allocate memory storage.
	
	\item The \acr{GUI} must initialize important information of the user such as name, age, address, gender, personal contact, and contact numbers of the responder with their name.
	
	\item After the device has detected that the user needs an assistance, the set responder/s will be notified through \acr{SMS}. A False alarm button is included in the device so that user has an option to stop the assistance.
	
	\item Fall detection is the focus of this project. The hip and thigh will have an individual accelerometer to determine basic body orientations such as sitting, standing, walking and lying.
	
\end{enumerate}

\subsection{Delimitations}
\begin{enumerate}
	\item The communication ability of the device will only be limited to the Philippines.
	
	\item \acr{SMS} are system-generated. The user cannot modify nor manually send  SMS however the user can interrupt SMS that will be sent.  
	
	\item The current location of the user will be retrieved as GPS coordinates and not as the exact address of the user.
	
	\item Two accelerometers will be used for the project, one for the upper body and one for the lower body. Two accelerometers are being used in accordance to the effectivity of the same number of sensors from the past related studies. By these, limited motions and body orientation of the user will be detected.
	
\end{enumerate}

\section{Description of the \documentType}

This project is a device that can detect if a user had a sudden fall and the direction on how the user had fallen. It can detect fall directions like forward, backward and sideward. Other motions such as sitting, standing, walking and lying can also be detected as a sub-feature of the device. It is user-friendly because the device can be set by accessing the \acr{GUI}. This device is designed to detect pre-fall, during the fall and post-fall motions. It can detect pre-fall motions which the user is possibly doing before the fall like sitting, standing or walking. It can also recognize whether the user had fallen and post-fall detection can also be acknowledged in which after the fall GPS coordinates.

When the device detects the said fall related incidents, it will also send a notification through \acr{GSM} communication to responders like family members, friends, and other people that the user nominates. The notification also includes the \acr{GPS} coordinates of the user, with the help of GPS module. 

\begin{figure}[htbp]
	\centering
		\includegraphics[width=0.50\textwidth]{"figure/Device-Illustration".jpg}
	\caption{Location of the device on the user}
	\label{fig:Device-Illustration}
\end{figure}

Shown in Figure 1.1 is the way the device will be worn by the user. The Main Board will be located at the hip of the user and can be attached on its belt. The Main Board has a built-in accelerometer. GSM Module and SD Card Module will also be attached on the Main Board. The External Accelerometer will be placed on the thigh of the user which can be attached on a belt or a garter.

\begin{figure}[htbp]
	\centering
		\includegraphics[width=1.00\textwidth]{"figure/GUI/Setup/1".PNG}
	\caption{Initial User Setup}
	\label{fig:1}
\end{figure}

\begin{figure}[htbp]
	\centering
		\includegraphics[width=1.00\textwidth]{"figure/GUI/Setup/2".PNG}
	\caption{User Setup, logging in and password setting}
	\label{fig:2}
\end{figure}

\begin{figure}[htbp]
	\centering
		\includegraphics[width=1.00\textwidth]{"figure/GUI/Setup/3".PNG}
	\caption{Profile Setup}
	\label{fig:3}
\end{figure}

\begin{figure}[htbp]
	\centering
		\includegraphics[width=1.00\textwidth]{"figure/GUI/Setup/4".PNG}
	\caption{Responder's Setting}
	\label{fig:4}
\end{figure}

\begin{figure}[htbp]
	\centering
		\includegraphics[width=1.00\textwidth]{"figure/GUI/Setup/5".PNG}
	\caption{Confirmation of Identification and password}
	\label{fig:5}
\end{figure}

\begin{figure}
	\centering
		\includegraphics[width=1.00\textwidth]{"figure/GUI/Setup/6".PNG}
	\caption{Setup success page}
	\label{fig:6}
\end{figure}

Figure 1.2 to figure 1.7 shows the GUI interface of the device set-up. Through these windows, the user’s information will be recorded wherein the said data may only be edited or changed by the user. Through this setup, responder/s can only view the data recordings. As shown in Figure 1.2, the simplicity of the interface can be seen as the device is being initialized, which makes the GUI user-friendly. In figure 1.3, the device is being setup by asking for a username and a password. Profile set up on Figure 1.4 initialized the user’s details. In this tab the complete name, address where they are staying, contact number and gender is being registered. In Figure 1.5, the profile of the responder/s is being setup which includes the name and immediate contact (mobile) number. Confirmation of username and password can be seen in Figure 1.6. Lastly, the success page in Figure 1.7 is shown which concludes the user’s setup. Only the user has the access to override and change the data from the device. Responders have limited access on the device where the data has been recorded. 

\begin{figure}[htbp]
	\centering
		\includegraphics[width=1.00\textwidth]{"figure/GUI/Selecting-a-card-2".PNG}
	\caption{Storage Selection}
	\label{fig:Selecting-a-card-2}
\end{figure}

After the device has been setup, the user must first insert the SD card to the device to gather and save data. The user or responder can review the data by inserting the said SD card in a computer then launching the GUI. The GUI will first ask the user or responder to select the proper storage device as shown in Figure 1.8.

\begin{figure}[htbp]
	\centering
		\includegraphics[width=1.00\textwidth]{"figure/GUI/Explore/1-Profile_Tab".PNG}
	\caption{Profile Tab}
	\label{fig:1-Profile_Tab}
\end{figure}

\begin{figure}[htbp]
	\centering
		\includegraphics[width=1.00\textwidth]{"figure/GUI/Explore/2-Responders_Tab".PNG}
	\caption{Responder's Tab}
	\label{fig:2-Responders_Tab}
\end{figure}

\begin{figure}[htbp]
	\centering
		\includegraphics[width=1.00\textwidth]{"figure/GUI/Explore/3-Activity_Log".PNG}
	\caption{Activity Log}
	\label{fig:3-Activity_Log}
\end{figure}

\begin{figure}[htbp]
	\centering
		\includegraphics[width=0.50\textwidth]{"figure/GUI/Explore/prompt".PNG}
	\caption{Prompt}
	\label{fig:prompt}
\end{figure}

Shown in Figures 1.9 to 1.11 is the explore window where the data records can be found. Figure 1.9 shows the profile tab which includes the name, gender, contact number, and address of the user. Responders tab can be seen in Figure 1.10, from which the responder’s details such as name, and contact number can be seen. On the next tab shown in Figure 1.11, the activity log where the activities and motions of the user can be found. These tabs from the explore window can be seen by all but only  the user has the access to edit the recorded data. If the user wishes to modify some data, a prompt will show up as shown in Figure 1.12. The prompt will request the user’s username and password to continue the modification of the recorded data.

\section{Estimated Work Schedule and Budget}

\begin{figure}[H]
	\centering
		\includegraphics[width=0.95\textwidth]{"figure/Tables as Images/Work Sched1".png}
	\caption{Estimated Work and Schedule (a)}
	\label{fig:Work Sched1}
\end{figure}

\begin{figure}[H]
	\centering
		\includegraphics[width=0.95\textwidth]{"figure/Tables as Images/Work Sched2".png}
	\caption{Estimated Work and Schedule (b)}
	\label{fig:Work Sched2}
\end{figure}

\begin{figure}[H]
	\centering
		\includegraphics[width=0.95\textwidth]{"figure/Tables as Images/Work Sched3".png}
	\caption{Estimated Work and Schedule (c)}
	\label{fig:Work Sched3}
\end{figure}

\begin{center}
{\scriptsize
\begin{tabularx}{\textwidth}{p{0.3\textwidth}|p{0.25\textwidth}|p{0.25\textwidth}|p{0.1\textwidth}}
\caption{Estimated Budget} \label{Estimated Budget} \\
\hline 
\hline 
\textbf{Components and Parts} & 
\textbf{Basic Details} &
\textbf{Significance of components to the project} &
\textbf{Price} \\ 
\hline 
\endfirsthead
\multicolumn{4}{c}%
{\textit{Continued from previous page}} \\
\hline
\hline 
\textbf{Components and Parts} & 
\textbf{Basic Details} &
\textbf{Significance of components to the project} &
\textbf{Price} \\  
\hline 
\endhead
\hline 
\multicolumn{4}{r}{\textit{Continued on next page}} \\ 
\endfoot
\hline 
\endlastfoot
\hline

\begin{center}
Genuino101
\includegraphics[width=0.3\textwidth]{"figure/Parts/Genuino101".jpg} 
\end{center}
& This is a 32-bit microcontroller, powered by Intel Curie microprocessor that has 196 kb flash memory and 24 kb sram with built in Bluetooth LE and 6-axis accelerometer/gyro technology. \cite{Int18} %(Intel, 2018)
& This microcontroller will be used to detect and assess the motion and orientation of the user using its built-in accelerometer. 
& P 2,048.00\\

\hline

\begin{center}
GSM/GPS/GPRS/SMS SIM808 CROWTAIL
and Sim Card
\includegraphics[width=0.3\textwidth]{"figure/Parts/GSM GPS Module".jpg} 
\end{center}
& This is a GSM and GPS two-in-one function module which is called Crowtail- SIM808.  It is very small and based on the latest GSM/GPS module SIM808 from SIMCOM, it supports GSM/GPRS Quad-Band network and combines GPS technology for satellite navigation. It has high GPS receive sensitivity with 22 tracking and 66 acquisition receiver channels that will lets you add location-tracking, voice, and text. \cite{Cir18} %(Circuitrocks, 2018)
& This GSM/GPS/GPRS module will be used to detect the GPS coordinates of the user. It has also GSM capabilities, so that we can send SMS message to notify responder/s that the user need help and assistance.
& P 1,917.00
P 50.00\\

\hline

\begin{center}
SD card socket
Module and 
SD card
\includegraphics[width=0.3\textwidth]{"figure/Parts/From Internet/MicroSd Module".jpg} 
\end{center}
& This is an Arduino compatible module and SD card library could be used with this device. \cite{Cir181} %(Circuitrocks, 2018)
& The SD card module will be used to read and write to the SD card to create a database. The database will be composed of data collected and the basic information of the user.
& P 105.00
P 300.00\\

\hline

\begin{center}
MPU6050 Accelerometer Module
\includegraphics[width=0.2\textwidth]{"figure/Parts/From Internet/MPU6050 Module".jpg} 
\end{center}
& This is an InvenSense MPU-6050 sensor that contains a \acr{MEMS} accelerometer and MEMS gyro in a single chip. It contains 16-bits analog to digital conversion hardware for each channel. It captures the x, y, and z channel at the same time. \cite{Cir182} %(Circuitrocks, 2018)
& This module will be used to detect falling motions and the specific motions of the user. Together with the built-in accelerometer in the Arduino 101 module, acceleration and dynamics can be computed which can be used to analyze the said motions.
& P 162.00 \\

\hline

\begin{center}
Piezo Buzzer
\includegraphics[width=0.3\textwidth]{"figure/Parts/From Internet/Speaker".jpg} 
\end{center}
& This is a piezoelectric speaker or buzzer that uses the piezoelectric effect for generating sound. It is a d36 mm Mylar Cone 8ohms 0.5w \cite{eGi18} %(e-Gizmo Mechatronix Central, 2018)
& This component will be used to alarm as it notifies the user when the device detects a falling motion or to alarm the user if the memory of the device is almost full.
& P 22.00 \\

\hline

\begin{center}
Push Button
\includegraphics[width=0.2\textwidth]{"figure/Parts/From Internet/False Alarm Button".jpg} 
\end{center}
& This is a side tact switch SPST through hole 7x4mm package/image: 77008044. It causes a temporary change in the state of an electronic circuit only while the switch is physically actuated. \cite{eGi181} %(e-Gizmo Mechatronix Central, 2018)
& This push button is used to terminate the process of the device on sending SMS to the responder recorded.
& P 6.00\\

\hline

\begin{center}
9v Rechargeable Battery
\includegraphics[width=0.3\textwidth]{"figure/Parts/9V Battery".jpg} 
\end{center}
& This is a 9-volt rechargeable battery with 300 mAh which can be charged into standard charger for 14-15 hours.
& This battery will be used to power up the fall analyzer device.
& P 493.50\\

\hline

\begin{center}
9v Battery Snap
\includegraphics[width=0.3\textwidth]{"figure/Parts/Battery Clip".jpg} 
\end{center}
& This is a battery clip for 9v battery attachment with complete red and black power lines. \cite{eGi182} %(e-Gizmo Mechatronix Central, 2018)
& This battery snap will be used for connecting the device to the 9v battery that will serve as a power source. 
& P 5.00 \\

\hline

\begin{center}
Power Switch
\includegraphics[width=0.2\textwidth]{"figure/Parts/From Internet/Power Switch".jpg} 
\end{center}
& This is a 6A 250V AC/10A 125 Ac Snap boat rocker switch \cite{eGi183} %(e-Gizmo Mechatronix Central, 2018)
& This switch will be used in connecting the power source to the device.
& P 15.00 \\

\hline

\begin{center}
300-ohm resistor
\includegraphics[width=0.25\textwidth]{"figure/Parts/From Internet/300 ohm resistor".jpg} 
\end{center}
& This is a 300-ohm resistor 1/4W. \cite{eGi185} %(e-Gizmo Mechatronix Central, 2018)
& This resistor will be used for controlling the voltage and current going to the LED
& P 0.25 \\

\hline

\begin{center}
Red LED
\includegraphics[width=0.25\textwidth]{"figure/Parts/From Internet/Red LED".jpg} 
\end{center}
& This is a 10-mm red and yellow light emitting diode. \cite{eGi184} %(e-Gizmo Mechatronix Central, 2018)
& This LED will indicate if the device is already connected to the GPS.
& P 6.50/pc \\

\hline

\begin{center}
Yellow LED
\includegraphics[width=0.25\textwidth]{"figure/Parts/From Internet/Yellow LED".jpg} 
\end{center}
& This is a 10-mm red and yellow light emitting diode. \cite{eGi184} %(e-Gizmo Mechatronix Central, 2018)
& This LED will indicate if the device is ready to use.
& P 6.50/pc \\

\end{tabularx}
}
\end{center}
\end{center}

\section{Overview of the \documentType}

The background of the study, the related studies, the statement of the problem, the general and specific objectives, the significance of the study and the scope and limitations of the study was overviewed in this chapter. On Chapter 2, the compilation of previous studies can be seen. It is a collection of past studies and how those studies helped in assembling and formulating of this study. Chapter 3 shows the theories where this project was based. In this chapter, the theoretical background on how each part of the fall analyzer was made. Chapter 4 contains the design of the project. Chapter 5 enumerates the methods on how the prototype of fall analyzer has been made. Chapter 6 contains the result and discussion of the experiment. In here the experiment’s data and the evaluation of those data can be seen. Lastly, Chapter 7 has the conclusion and recommendation of the study.