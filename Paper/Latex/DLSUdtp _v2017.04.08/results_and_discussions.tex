\section{Direction of Fall Output}

Falling motion is the main concern of this project. With the falling motion, detection of fall and its direction are to be considered to call the device a fall analyzer. From the experiment’s data, gathered over 30 respondents, participants are asked to do a falling motion with three classifications of fall based on the direction namely: forward fall, side fall and backward fall. Results are shown from the table 6.1

% Table generated by Excel2LaTeX from sheet 'Sheet2'
\begin{table}[htbp]
  \centering
  \caption{Falling Motions Data}
	\resizebox{0.5\textwidth}{!}{
    \begin{tabular}{|c|c|c|c|}
    \toprule
    \multirow{2}[4]{*}{Sample \#} & \multicolumn{3}{c|}{Falling Motions} \\
\cmidrule{2-4}          & Forward & Backward & Sideward \\
    \midrule
    1     & Lying & /     & / \\
    \midrule
    2     & /     & /     & / \\
    \midrule
    3     & /     & /     & / \\
    \midrule
    4     & /     & /     & / \\
    \midrule
    5     & /     & /     & / \\
    \midrule
    6     & /     & /     & / \\
    \midrule
    7     & /     & /     & / \\
    \midrule
    8     & /     & /     & / \\
    \midrule
    9     & /     & /     & / \\
    \midrule
    10    & /     & Lying & / \\
    \midrule
    11    & /     & /     & / \\
    \midrule
    12    & /     & /     & / \\
    \midrule
    13    & /     & Lying & / \\
    \midrule
    14    & /     & /     & / \\
    \midrule
    15    & /     & /     & / \\
    \midrule
    16    & /     & /     & / \\
    \midrule
    17    & /     & /     & / \\
    \midrule
    18    & /     & /     & Backward Fall \\
    \midrule
    19    & /     & /     & Lying \\
    \midrule
    20    & /     & /     & / \\
    \midrule
    21    & /     & /     & Lying \\
    \midrule
    22    & /     & /     & / \\
    \midrule
    23    & /     & /     & / \\
    \midrule
    24    & /     & /     & / \\
    \midrule
    25    & /     & /     & / \\
    \midrule
    26    & /     & /     & Lying \\
    \midrule
    27    & /     & /     & / \\
    \midrule
    28    & /     & /     & / \\
    \midrule
    29    & /     & /     & / \\
    \midrule
    30    & /     & /     & / \\
    \bottomrule
    \end{tabular}%
		}
  \label{tab:Falling Motions Data}%
\end{table}%

From Table 6.1, the device shows that it can detect fall and the direction of fall. Within 30 participants, the device shows that it can detect forward fall with 96.66\% accuracy that resulted from a 29 out of 30 accurate forward fall results. On the other hand, backward fall resulted in 93.33\% of backward fall accuracy detection and 86.67\% for side fall accuracy detection. Overall fall detection gives a reliability of 92.22\% accuracy in detection such motion.

The data shows that the device can clearly recognize forward fall with only 3.33\% error. This is because the device recognizes only such direction and motion that does not have any other alike motion that insist a moving forward action. The backward fall follows with 6.67\% rate of error. The device sometimes recognizes backward fall as just a lying motion if the user fell in a slow acceleration. As lying position induces a backward motion, the acceleration of the user will be the only boundary that separates the two motions in which if the device recognizes motions that are exactly in between the two, the 6.67\% error of backward fall motion comes to reason. Lastly, sideward fall came with 13.33\% error which has the highest error among the three falling motions. This is because of the device’s sensitivity in recognizing the fall. As the user falls  sideward, the device should recognize a sideward fall motion however as the user hits the ground, the algorithm computes the last position of the user as they fell basing from the initial position before they fell. In such case, if the user falls sideways, but hits the ground backside, the device recognizes a backward fall motion because of the last position of the user is will be in a slight facing up position. This makes the side fall detection the most sensitive of the three directions due to the position to consider after the user has fallen.

\section{Motion Detection Data}

Detection of other motions other than the falling motion is a sub-feature of this device. These motions to be detected are divided into two categories which are static motions and dynamic motions. Static motions such as sitting, standing and lying can be detected by this device so as the walking motion under the dynamic category. Table 6.2 shows the result of the device's capability on recognizing such motions with experimenting the device’s ability to identify the mentioned specific motions with 30 participants as a subject.

% Table generated by Excel2LaTeX from sheet 'Motion Detection Data'
\begin{table}[htbp]
  \centering
  \caption{Motion Detection Data}
	\resizebox{0.6\textwidth}{!}{
    \begin{tabular}{|c|c|c|c|c|}
    \toprule
    \multirow{2}[4]{*}{Sample \#} & \multicolumn{3}{c|}{Static Motions} & Dynamic Motion \\
\cmidrule{2-5}          & Standing & Sitting & Lying & Walking \\
    \midrule
    1     & /     & /     & /     & / \\
    \midrule
    2     & /     & /     & /     & / \\
    \midrule
    3     & /     & /     & /     & / \\
    \midrule
    4     & /     & /     & Walking & / \\
    \midrule
    5     & /     & /     & /     & / \\
    \midrule
    6     & /     & /     & /     & / \\
    \midrule
    7     & /     & /     & /     & / \\
    \midrule
    8     & /     & /     & /     & / \\
    \midrule
    9     & /     & /     & Backward Fall & / \\
    \midrule
    10    & /     & /     & /     & / \\
    \midrule
    11    & /     & /     & Standing & / \\
    \midrule
    12    & /     & /     & /     & / \\
    \midrule
    13    & /     & /     & /     & / \\
    \midrule
    14    & /     & /     & /     & / \\
    \midrule
    15    & /     & /     & /     & / \\
    \midrule
    16    & /     & /     & /     & / \\
    \midrule
    17    & /     & /     & /     & / \\
    \midrule
    18    & /     & /     & /     & / \\
    \midrule
    19    & /     & /     & /     & / \\
    \midrule
    20    & /     & /     & /     & / \\
    \midrule
    21    & /     & /     & /     & / \\
    \midrule
    22    & /     & /     & /     & / \\
    \midrule
    23    & /     & /     & /     & / \\
    \midrule
    24    & /     & /     & /     & / \\
    \midrule
    25    & /     & /     & /     & / \\
    \midrule
    26    & /     & /     & /     & / \\
    \midrule
    27    & /     & /     & /     & / \\
    \midrule
    28    & /     & /     & /     & / \\
    \midrule
    29    & /     & /     & /     & / \\
    \midrule
    30    & /     & /     & /     & / \\
    \bottomrule
    \end{tabular}%
		}
  \label{tab:Motion Detection Data}%
\end{table}%


From table 6.2, specific motions data such as standing, sitting, lying and walking can be seen. A 100\% reliability rate was achieved by the device in recognizing motions such as standing, sitting and walking. Because these motions do not interfere with any other motions that this device can detect, the complete success rate on motion recognition was achieved, however in detecting the lying motion, only 90\% accuracy rate was reached. This is because the device sometimes recognizes lying motion into a falling motion or in other cases it stops on the detection of the previous motion before the lying motion, just like the result on the table shows.

\section{Confusion Matrix}

Overall, falling and motion detection data can be seen in table 6.3. The table shows the confusion matrix of all motions including the direction of the falling motion. From the table, it shows that the overall reliability of the device in detecting motions such as walking, standing, sitting, lying, forward fall, backward fall and side fall is 95.23\% which makes the error of the device be only at 4.76\% only.


% Table generated by Excel2LaTeX from sheet 'Sheet1'
\begin{table}[htbp]
  \centering
  \caption{Confusion Matrix}
	\resizebox{1.00\textwidth}{!}{
    \begin{tabular}{|c|c|c|c|c|c|c|c|c|c|}
\cmidrule{7-9}    \multicolumn{1}{c}{} & \multicolumn{1}{c}{} & \multicolumn{1}{c}{} & \multicolumn{1}{c}{} & \multicolumn{1}{c}{} &       & \multicolumn{3}{c|}{Falling} & \multicolumn{1}{c}{} \\
\cmidrule{2-9}    \multicolumn{1}{c|}{} &       & Walking & Standing & Sitting & Lying & Forward & Backward & Side  & \multicolumn{1}{c}{} \\
\cmidrule{2-9}    \multicolumn{1}{c|}{} & Walking & 30 & 0     & 0     & 1     & 0     & 0     & 0     & \multicolumn{1}{c}{} \\
\cmidrule{2-9}    \multicolumn{1}{c|}{} & Standing & 0     & 30 & 0     & 0     & 0     & 0     & 1     & \multicolumn{1}{c}{} \\
\cmidrule{2-9}    \multicolumn{1}{c|}{} & Sitting & 0     & 0     & 30 & 1     & 0     & 0     & 0     & \multicolumn{1}{c}{} \\
\cmidrule{2-9}    \multicolumn{1}{c|}{} & Lying & 0     & 0     & 0     & 27 & 1     & 2     & 4     & \multicolumn{1}{c}{} \\
\cmidrule{1-9}    \multirow{3}[6]{*}{Falling} & Forward & 0     & 0     & 0     & 0     & 29 & 0     & 0     & \multicolumn{1}{c}{} \\
\cmidrule{2-9}          & Backward & 0     & 0     & 0     & 1     & 0     & 28 & 1     & \multicolumn{1}{c}{} \\
\cmidrule{2-10}          & Side  & 0     & 0     & 0     & 0     & 0     & 0     & 26 & Overall Reliability \\
    \midrule
    \multirow{2}[4]{*}{Reliability} &       & 1     & 1     & 1     & 0.9   & 0.966667 & 0.933333 & 0.866667 & 0.952380952 \\
\cmidrule{2-10}          & in Percentile & 100\% & 100\% & 100\% & 90\%  & 97\%  & 93\%  & 87\%  & 95\% \\
    \bottomrule
    \end{tabular}%
		}
  \label{tab:Confusion Matrix}%
\end{table}%

\section{Notification System}

As the user fell, a notification system was activated. Using GSM technology, an SMS was sent into the recorded responder/s. In the message, the name of the user, location where the fall happened, date of fall and time of fall were included into the notification message.

\begin{figure}[htbp]
	\centering
		\includegraphics[width=0.5\textwidth]{"figure/Data/Notifiation Message".png}
	\caption{Notification Message}
	\label{fig:Notifiation Message}
\end{figure}

Figure 6.1 shows the notification messages to two different responders. Responder 1 uses sim1 and responder 2 uses sim 2. As can be seen from the image above, the device has sent the user’s name and details such as location, date and time after he fell. Notice that the device sent the message to the two responders approximately 1 minute after the fall. The location of fall was presented into coordinates and on Figure 6.2, the sent coordinates can be found on the google maps to be located at De La Salle University – Manila Laguna Campus. However, when the user has fallen and decided to hit the false alarm button in 10 seconds, the message notification system was override and message was not sent.

\begin{figure}[htbp]
	\centering
		\includegraphics[width=1.00\textwidth]{"figure/Data/Location of Fall from GOogle Maps".png}
	\caption{Location of Fall from Google Maps}
	\label{fig:Location of Fall from GOogle Maps}
\end{figure}
