\section{Summary}

In this chapter, the hardware and software consideration in making the fall analyzer device can be seen. This chapter explains the parts and functions of fall analyzer device.

\section{Fall Detection}

To detect fall, instruments and devices are created to identify if the observed object undergoes the action of fall. In human detection of fall, considerations such as the direction of fall, the sections of the body that can help to consider fall, the types of fall and the way how to analyze fall must be considered.

\subsection{Motion Detection and Direction of Fall}

The orientation of the fall is one of the components on analyzing fall. The object’s fall direction whether it is falling or had fallen in the directions such as forward, backward or sideward: right or left, is an important factor to identify fall. With the use of the triaxial-accelerometer, the direction of fall can be identified. The tri-axial accelerometer can acknowledge orientations with respect to the earth’s gravity.

Specific motions can also be acknowledged by this device. Static motions such as sitting and standing and dynamic motions such as walking and falling are motions that this device can evaluate. Using the same module, the accelerometer can measure motions. Static and dynamic motions can be recognized by the accelerometer thru analyzing the consistency and rapidness of the changes in motions. 

\subsection{Location of the User}

The location of the user can be known using the GPS module which is included in this device. The user’s location is being recorded each time as the device is together with the user. After the fall, the coordinates of the user where s/he has fallen is included on the data to be sent to the defined responder/s so that as the responder/s receives the notification, the user can be easily located. Using the ability of the GPS module to identify the exact coordinate, time and date, added information as such can be helpful to the responder in dealing with the fall.

\subsection{Alert System}

Using the GSM module, alert system via SMS is being implemented. At the moment the user has fallen, and the device has recognized the fall, using the said module  information and notification alert will be sent to the defined responder/s. On the other hand, when the user has fallen and chooses not to initiate the alert system, it can be overriden by pressing a specific button that cancels the alert. This is done to avoid false alarm and to give the user a choice whenever the fall does not result to a heavy damage.

\subsection{Graphical User Interface (GUI)}

The GUI makes the personalization setting or initialization of the device - which includes the information of both the user and responder as well as the retrieval of recorded activities or motions before, during and after the fall - easy. With the use of the GUI, the initiating and retrieval became user friendly where the recorded data can be easily found and understood.

\subsection{Sections of the Body}

Another factor to consider in fall detection is the sections of the body. These must be considered to identify which part of the body does the sensors better detect motions such as walking, sitting, standing, lying and most importantly falling. To detect fall, sensors are introduced into specific areas of the body, particularly the hip and thigh. Through these parts of the human body, motions mentioned above can be easily analyzed because of the specific movement that these two create whenever the subject is moving.

\begin{figure}[htbp]
	\centering
		\includegraphics[width=0.4\textwidth]{"figure/Proper Wear".jpg}
	\caption{Device on the user}
	\label{fig:Proper Wear}
\end{figure}

\subsection{Trip, Stumble, and Fall}

In many descriptions, fall can be defined as the moment of having tripped, stumbled, then fell. The device which was created to recognize falls detects all of these descriptions if the subject fell flatly on the ground after such events. Generally, those descriptions mentioned above became a sub-definition of fall. Tripping and stumbling will not be recognized if a fall does not occur after those scenarios.

\subsection{Vector Analysis of Fall}

From the sensor’s data, not only the direction of fall or fall itself will be analyzed. Through vector analysis of the data, other motions such as standing, sitting walking and lying will also be analyzed as a sub-feature of the device. The magnitude of acceleration follows (3.1), where “a” stands for the acceleration while “x, y and z” are values detected by the tri-axial accelerometer.

Vector of cosines (3.2 to 3.4), are used to detect walking. The angles are used to determine the difference of the wideness of the swing of the leg during the said motion. As two types of motions have been categorized as static and dynamic motions where they are defined as motions that are in constant pattern and motions that are rapidly changing respectively, averaging algorithm (3.5) is needed to accurately predict and determine static motions in which will be the basis in identifying dynamic motions as well. Lastly, errors must be computed (3.6) to determine the reliability of the measured data.

\begin{eqnarray}  
a^2 = x^2 + y^2 + z^2
\end{eqnarray}

\begin{eqnarray}
cosx\theta = \frac{x}{magnitude} 
\end{eqnarray}

\begin{eqnarray}
cosy\theta = \frac{y}{magnitude} 
\end{eqnarray}

\begin{eqnarray}
cosz\theta = \frac{z}{magnitude} 
\end{eqnarray}

\begin{eqnarray}
Statistic Value - \frac{Mean}{Average}
\end{eqnarray}

\begin{eqnarray}
\left| \frac{Present - Past}{Present} x 100 \right|
\end{eqnarray}

\section{Pre, During and Post Detection of Fall}

The device will analyze data before, during and after the fall happens. This will be recorded and will be used to analyze data.

\subsection{Pre-fall Detection}

On the pre-fall evaluation, the device will analyze the movement of the user before the fall happens. These movements can be classified as standing, walking, sitting and lying. These data can be used as a sub-feature of the device and can be used by the responder/s to know the previous activity of the user before the fall happened.

\subsection{Post Fall Detection}
After the fall, the device will send emergency messages to the responder/s that are recorded in the database. The device will send the location and of the user where the subject has fallen and the time they fell.