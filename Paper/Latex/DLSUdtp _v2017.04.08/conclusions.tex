\section{Concluding Remarks}

Designing a device that can detect motion of user such as standing, sitting, lying, walking and falling with direction detection of fall whether the fall is in forward, sideward or backward motion was successfully developed with 95\% of the device’s accuracy of detection. When falling motion was established, the notification system of the device was activated, and message was sent to the recorded responder/s with details such as location of fall, time of fall, name of user who has fallen and the date of fall.  

The use of two accelerometers located at the hip and thigh resulted as a good component in making a motion detection device with 92.22\% reliability. The development of falling device that can recognize direction of fall with the use of the same two modules resulted with 92.22\% of accuracy which supports the conclusion of the detection of the user’s orientation of fall whether the user has fallen backward, forward or sideward to be achieved. The GPS module helps to get the time and date of the user’s fall.


Time, date, location, and motions that were detected since the device was activated were also saved on a database through SD Card. Stored data were retrieved through the GUI that interacts with the user as it was created with user-friendly functions and directions. 

Motion detection such as standing, sitting, walking and lying that are sub-feature of this device was successfully detected with 97.5\% overall accuracy rate. As a general conclusion, the design and development of fall analyzer with specific motion detection which can be personalized with the function of send notification message to recorded responder/s during the fall was achieved with above 80\% acceptable accuracy rate.

\section{Contributions}

The synthesis of all the contributions that this thesis has made and developed are as follows:

\begin{itemize}
  \item Fall detection with the recognition of fall direction whether backward, sideways or forward.
	
	\item Detection of motions such as standing, sitting, lying and walking with data logging of such motions which can be accessed by the responder/s thru GUI.
  
  \item Notification and alarm system via GSM with GPS location, and time data included on the responder’s notification message.
	
\end{itemize}


\section{Recommendations}

For future innovation of the study, detection of motions along a not plain terrain like stairs and alike is recommended. Power management of the device must also be improved for the further improvement of this study as this project limits itself on the detection. The location of the user when fallen in this project can be identified using the GPS coordinates, so for further study, the use of exact address in locating the user is suggested. Lastly, a training process of the device’s system can be added for better detection of motions for every different user.

\section{Future Prospects}

In further accretion of this device, the use of fall and motion detection can be used in medical field. This can serve as a monitoring device to patients like the elderly who might have suffer fall related injuries. Using this technology, accidents related to fall can be prevented and can be taken care of as quickly as possible. Using the real-time notification system with the included time, location and might include the previous activity of patient before fall, a monitoring device such this can be helpful. 

Another application for future use of this device is for manpower activity detection. Due to the location register and activity log of the device, it can detect the activity of the user e.g. in the shopping mall, the device can be used by the salesperson and the manager can detect whether his/her employee is working or in his/her assigned work location.

