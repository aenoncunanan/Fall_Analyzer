\section{Existing Work}

For aged population, falls are dangerous for they can adversely affect health. \cite{LiQ091} %(Li, Hanson, Stankovic, Barth, & Lach, 2009)
Fracture, certain possibility to get coma, brain trauma, and paralysis are the most common injury in fall of an elderly. The high impact is the main source of injury at most fall situations. But sometimes the late medical salvage may worsen the situation. That means the faster the salvage comes, the less risk the elderly will face which makes the progress of technology brings more possibilities to help us protect the elderly \cite{WuF14} %(Wu, Zhao, Zhao, & Zhong, 2014).

Several kinds of fall detection methods have been developed or applied in our life. Most of the research on falls where accelerometers are used focus on determining the change in magnitude of acceleration. When the acceleration value exceeds a critical threshold, the fall is detected. These systems successfully detect falls with sensitivities greater than 85\% and specificities between 88-94\%. However, focusing on large acceleration only can result in many false positives from fall-like activities such as sitting down quickly and running. \cite{Huy13} %(Huynh, Nguyen, Tran, Nabili, & Tran, 2013).

Some fall detection algorithms also assume falls happen when the body lies prone on the floor which are less effective when a person’s fall posture is not horizontal, e.g. fall happened on stairs. Furthermore, previous studies used complex algorithms like \acr{SVM} and Markov models to detect the fall. However, accuracy of these systems has not been proven to be highly effective. They also use excessive amounts of computational resources and cannot respond in real time. In addition, fall activity patterns are particularly difficult to obtain for training systems. Unlike other previous research, this project proposes using the accelerometer sensor to detect the falls for increasing the sensitivities and specificities of a fall detection system. Accelerometers can detect orientation and axis of movements as it is one of the feature of these technologies \cite{Huy13} %(Huynh, Nguyen, Tran, Nabili, & Tran, 2013).

\noindent \textbf{\textit{Fall detection:}}

Existing fall detection solutions can be divided into two classes. The first class only analyzes acceleration to detect falls. A second class of solutions utilize both acceleration and body orientation information to detect falls \cite{LiQ091} %(Li, Hanson, Stankovic, Barth, & Lach, 2009).
The algorithm uses the accelerometer present in cellphones to monitor falls. If a fall is detected, the application automatically notifies predefined contacts (such as parents or emergency services) with the victim’s GPS coordinates shown on a map. \cite{Kaz14} %(Kazi, Sikander, & Yousafzai, 2014)

When the subject falls, the acceleration is rapidly changing, and the angular velocity produces a variety of signals along fall direction. The lower and upper fall thresholds for the acceleration and angular velocity used to identify the fall are derived as follows:

\begin{itemize}
	\item \acr{LFT}: the negative peaks for the resultant of each recorded activity are referred to as the \acr{LPV}. The LFT for the acceleration signals are set at the level of the smallest magnitude \acr{LFP} recorded.

	\item \acr{UFT}: the positive peaks for the recorded signals for each recorded activity are
referred to as the signal \acr{UPV}. The UFT for each of the acceleration and the angular velocity signals were set at the level of the smallest magnitude UPV  recorded.  The UFT  is related to the peak impact force experienced by the body segment during the impact phase of the fall \cite{Huy13} %(Huynh, Nguyen, Tran, Nabili, & Tran, 2013). 
\end{itemize}

Like every other past studies, accelerometer and gyroscope have limited abilities to measure a fall.  All past algorithms involving these two components, had a good detection rate for actual falls, and had low false negatives except in the case of jumping onto a walking position in a bed or running. \cite{Kaz14} %(Kazi, Sikander, & Yousafzai, 2014)

\noindent \textbf{\textit{Transmission:}}

When a fall has certainly occurred, there is still one false positive that remains to be accounted for, that is if a user accidentally falls and can still manage to stand up or simply in cases where the device can call the incident a false alarm. To account for this, a 10-second ‘grace period’ has been added to the device for the user where the device delays sending a text to a responder or the user can choose to cancel altogether. If the alarm is not cancelled within a pre-specified time, the device will send the persons GPS coordinates with a message to predefined people via GSM.  \cite{Kaz14} % (Kazi, Sikander, & Yousafzai, 2014)

\noindent \textbf{\textit{Data acquisition:}}

In the hardware part, microcontroller handles the acquisition of accelerometer data, captures UTC time using an additional discriminator connected to the GPS receiver module and merges these data for transmission via Bluetooth module. To communicate with these modules, the microcontroller   uses communication   protocol   for accelerometer module and UART interface for GPS and Bluetooth® modules, respectively. Microcontroller program codes were written and compiled with Proton IDE software \cite{Gur14}. %(Gurkan, Gurkan, Dindar, Akpmar, & Gulal)
All data can be logged to memory card for responder/s acquisition.

\section{Lacking in the Approaches}

\begin{center}
{\scriptsize
\begin{tabularx}{\textwidth}{p{0.4\textwidth}|p{0.5\textwidth}}
\caption{Lacking Approaches} \label{Lacking Approaches} \\
\hline 
\hline 
\textbf{Existing Work} & 
\textbf{Lacking Approaches} \\ 
\hline 
\endfirsthead
\multicolumn{2}{c}%
{\textit{Continued from previous page}} \\
\hline
\hline 
\textbf{Existing Work} & 
\textbf{Lacking Approaches} \\ 
\hline 
\endhead
\hline 
\multicolumn{2}{r}{\textit{Continued on next page}} \\ 
\endfoot
\hline 
\endlastfoot
\hline

Accurate, Fast Fall Detection Using Gyroscopes and Accelerometer-Derived Posture Information \cite{LiQ091} %(Li, Stankovic, Hanson, Barth, \& Lach, 2009)
&
This study combines the use of gyroscope and accelerometer to create a device that can detect a fall -  which is different to other studies that uses accelerometer that isolates falls from \acr{ADL}. This study sets two categories that divides ADL, the static and dynamic postures. Motions such as standing, sitting and lying does not limit our study. Our study has a sub feature of determining a dynamic motion such as walking. Using the same modules, the accelerometer and gyroscope, determines whether the user is subjected to fall even when they are in static motion like sitting, standing and lying so as in dynamic motion like walking.

This study also uses linear acceleration and angular velocity to determine whether the motion transition is intentional and uses it to determine a sudden fall. Our study however, uses time difference of the previous position and the  current position when the user has fallen. Using this simple method, intentional fall can be differentiated with sudden fall and it uses a simple computational algorithm that can help the device to compute data easily compared to this study. \\

\hline

A Wireless Body Area Network of Intelligent Motion Sensors for Computer Assisted Physical Rehabilitation \cite{Jov05} %(Jovanov, Milenkovic, Otto, \& De Groen, 2005)
&
The focus of this study is to create a wireless sensor that can be used to detect multi-tier telemedicine system that can be implemented by the aid of computer-assisted physical rehabilitation applications and ambulatory monitoring. Our study focuses on the same concept but is specialized on the fall detection of a patient where in order for the device to detect a fall, it does not need the continuous attendance of a computer-based monitor nor the attendance of a person who will manually monitor the capability of device to monitor the overall medical performance of the user.

The communication system of this study that gives notification to the responder, and the ability to send the recorded data from the device rely on the use of internet, while our study submits notification to the responder using SMS and the recorded data of the user before, during and after the fall can be access directly through the device. This means the absence of internet does not interfere to the notification ability of the system of our project, especially the design was based from the Philippine settings in which internet connection is not reliable. \\

\hline

Enhancing the Quality of Life Through Wearable Technology \cite{Par03} %(Park \& Jayaraman, 2003)
&
Wearable technology is the focus of this study. It provides a wearable medical device that detects medical issues of user and give information to the attending guardian or doctor. It includes a motherboard that is based on fabric and was created to be a smart wearable shirt. The wearable characteristic of this study was achieved, and, in our study, we tried to create a device that can be carried by the user and can detect fall emergency that can also give details of the pre-fall, during fall and post-fall events of the user to the responder. The user-friendly feature, and the “can be carried device” characteristic can be considered the wearable feature of the device. Also, the information of the user as they use the device in our study was enhanced by the location detector, fall detector and the ability to send a message to a responder during a fall of the said device.   \\

\hline

Development of a Wearable-Sensor-Based Fall Detection System \cite{WuF14} %(Wu, Zhao, Zhao, \& Zhong, 2014)
&
This study used the accelerometer to identify, analyze and recognize fall. It comes together with a GPS/GSM module that is used only when the user is outdoor. Through these devices, the location of the user when fallen can be identified and can be used to notify the responder to give an immediate action for the incident. Our study also uses accelerometer. Though through the results of other studies it was proven that the combination of accelerometer and gyroscope increases the efficiency of identifying fall, in this study vector analysis was used in accelerometer to allow to have the module a better range of identifying motions. Using the tri-axial accelerometer, the previous activities of the user such as walking, standing or sitting can also be known before the user has fallen. On the other hand, the use of GPS/GSM module in our study was maximized. The location of the user can also be identified even when he/she is indoor and not just outdoor. An SMS is being sent in case the user has fallen anywhere and anytime to the registered responder from the module itself. \\

\hline

Posture and Movement Classification: The Comparison of Tri-Axial Accelerometer Numbers and Anatomical Placement \cite{Fot14} %(Fotune, Lugade, \& Kaufman, 2014)
&
The placement of tri-axial accelerometers on to the body was studied in this project. Three areas of the body were used to identify the most accurate results that an accelerometer can give. In this study, the waist, thigh and ankle were tested differently. As a result, for a single accelerometer, the thigh part is the best area that can give the better accurate result for the accelerometer. However, for two accelerometers, the combination of waist-thigh gives the most responsive and accurate results among other areas of the body. In our study, waist-thigh area is used to place the accelerometer and two accelerometers were also used. The combination of the built-in accelerometer from the genuino 101 and another accelerometer clearly identifies the static and dynamic motion of the user through analyzing its vector component. \\

\hline

Fall Detection System Using Combination Accelerometer and Gyroscope \cite{Huy13} %(Huynh, Nguyen, Tran, Nabili, \& Tran, 2013)
&
This study investigates the methodology to identify falls from normal \acr{ADL}. In this study, a wireless sensor system, based on accelerometer and gyroscope, is placed at the center of the chest to collect real time fall data. Experiment protocols consisting of four types of fall such as forward, backward and sideward (right and left) falls along with normal gait. \cite{Huy13} %(Huynh, Nguyen, Tran, Nabili, \& Tran, 2013).
Our study also identifies the direction of fall using the combination of two accelerometers, however these modules were not placed on the chest but rather on the waist and thigh as those were the best areas to locate the sensors based from the other studies that concerns the detection of fall and ADLs. In addition, the location on where the user has fallen is being identified in our study and in such case a notification system is activated via GSM that gives notification to the registered responder in the module. \\

\hline

Fall Detection Using Single Tri-Axial Accelerometer \cite{Kaz14} %(Kazi, Sikander, \& Yousafzai, 2014)
&
This study uses a single tri-axial accelerometer that is a mobile phone-based system which implements a fall detection algorithm using a mobile phone’s built-in accelerometer which can detect falls with the victim’s GPS coordinates displayed on the map for timely delivery of medical help. \cite{Kaz14}. %(Kazi, Sikander, \& Yousafzai, 2014).
Our study is similar with this -, it can also detect fall using a tri-axial accelerometer. The notification system also identifies the predetermined responders in case of fall together with the location of the user using GPS coordinates. In addition, our study also identifies and differentiate motions such as static and dynamic from fall. Motions such as standing, sitting, lying and walking can also be determined in our study and during the response of the responder in such incident of fall, the previous activity of the user can be identified and be used for further evaluation of the situation. \\

\hline

Wireless 3-Axis Accelerometer System for Measuring of Structural Displacement \cite{Gur14} %(Gurkan K. , Gurkan, Dindar, Akpinar, \& Gulal, Engin, 2014)
&
This study presents the design of a 3-axis acceleration measurement system that can Coordinate \acr{UTC} time-stamping captured from \acr{GPS} receiver module. Acquired 3-axis acceleration data and UTC are transferred via Bluetooth protocol and developed software which enables monitoring and recording UTC and acceleration data on PC, respectively. \cite{Gur14} %(Gurkan, Gurkan, Dindar, Akpinar, \& Gulal, Engin, 2014).
This study uses UTC to record time and GPS to analyze displacement combined with the help of acceleration to identify displacement of user. These combined project results to an acceptable result but with a huge delay due to the complications of calculations. In our study, time and location can be identified by the help of GPS module while displacements or the body movement of user and the body orientation is being analyzed using the two tri-axial accelerometers. Both sets of modules aims to identify the displacement, , time and body orientation but our study enhances the delay of this study. \\


\end{tabularx}
}
\end{center}



